\documentclass[11pt]{letter}
\usepackage{amsmath}
\usepackage{amssymb}
\usepackage{hyperref, url}
\usepackage[margin=1.5in]{geometry}
\begin{document}
\textbf{Math 161 - Midterm Topic} \\
\textbf{Due Friday, 24 March at 11.59 PM}

\textbf{Priya Malhotra}

I would like to prove Rolle's theorem. Rolle's theorem states that any real-valued differentiable function that attains equal values at two distinct points must has at least one stationary point somewhere between them, so the point where the first derivative is 0. 

More formally put, for a real-valued function $f$, which is continuous on a proper closed interval $[a,b]$, and differentiable on the open interval $(a,b)$ where $f(a) = f(b)$, then $\exists c \in (a,b)$ such that $f'(c) = 0$. 

The generalization of Rolle's theorem: For some real-valued continuous $f$ on a closed interval $[a,b]$, with $f(a) = f(b)$, if for every $x$ in the open interval $(a,b)$, the right hand limit $f'(x^+) := \lim_{h \rightarrow 0^+} \frac{f(x+h) - f(x)}{h}$ and the left hand limit $f'(x^-) := \lim_{h \rightarrow 0^-} \frac{f(x+h) - f(x)}{h}$ exist in the extended real line $[-\infty, \infty]$, then $\exists c \in (a,b)$ such that one of the two limits $f'(c^+)$ and $f'(c^-)$ is greater than 0 and the other is less than 0. If the right and left hand limits agree for every $x$, then they agree in particular for $c$, so the derivative of $f$ exists at $c$ and is equal to 0. 

The proof idea is that if $f(a) = f(b)$, then $f$ either reaches a maximum or minimum somewhere between $a$ and $b$, and at this point, the function changes from increasing to decreasing, or vice versa, and at this point, the derivative is 0. I will split my lean project into roughly these parts when solving it, and will incorporate information on derivatives, and general calculus rules. 



References:
https://en.wikipedia.org/wiki/Rolle\%27s\_theorem




\end{document}